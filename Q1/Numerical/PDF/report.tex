\documentclass{article}
\usepackage{ amssymb }
\usepackage{amsmath}
\usepackage[amsmath,amsthm,thmmarks]{ntheorem}
\usepackage{mathtools}
\usepackage{tikz}
\usepackage{multicol}
\usepackage[margin=0.8in]{geometry}

\title{Numerical methods for Differential equations Report 1}
\author{Malte Wegener}
\date{September 2019}

\begin{document}

\maketitle
%\begin{multicols}{2}
    
\section{Problem 1}
\subsection{a) Proof of Fermat's theorem}

A sufficiently smooth function $f(x)$ has a local maximum in $x_{0}$ if there exits $\delta > 0$ such that $f(x) \leq f(x_{0})$ for all $x, \mid x - x_{0} \mid > \delta$ \par
Therefore $f(x_{0}+\epsilon)-f(x_{0}) \leq 0$ for $\mid \epsilon \mid < \delta$
If we divide this by a small positive $\epsilon$, we get $\frac{f(x_{0}+\epsilon)-f(x_{0})}{\epsilon}\leq 0$.
Taking the limit as $\epsilon$ goes to 0, we get
\begin{equation*}
   \lim_{\epsilon\to 0^{+}} \frac{f(x_{0}+\epsilon)-f(x_{0})}{\epsilon}\leq 0
\end{equation*}

\begin{equation*}
   f'(x_{0})\leq 0
\end{equation*}

Similarly for $\epsilon < 0$

\begin{equation*}
   \lim_{\epsilon\to 0^{-}} \frac{f(x_{0}+\epsilon)-f(x_{0})}{\epsilon}\geq 0
  \end{equation*}
   
\begin{equation*}
   f'(x_{0})\geq 0
\end{equation*}

Thus $f'(x_{0}) = 0 \blacksquare$

\subsection{b) Converse of the second derivative test}
Let $f(x)$ be a function with a maximum in $x=x_{0}$ that can be expanded by a taylor series around $x_{0}$. f in the neighborhood of $x_{0}$ can than be described by the following Taylor series.
\begin{equation*}
    f(x_{0}+\delta) = f(x_{0})+\delta*f'(x_{0})+\frac{\delta^2}{2}*f''(x_{0}) + \mathcal{O}(\delta^3)
\end{equation*}
As $f'(x_{0}) = 0$, as proven in a), This series can be rearranged for $f''(x_{0})$.

\begin{equation*}
    f(x_{0}+\delta) - f(x_{0}) = \frac{\delta^2}{2}*f''(x_{0}) + \mathcal{O}(\delta^3)
\end{equation*}
 As $f(x_{0}+\delta) - f(x_{0}) \leq 0$ in the neigborhood of $x_{0}$.
\begin{equation*}
    0 \geq \frac{\delta^2}{2}*f''(x_{0}) + \mathcal{O}(\delta^3)
\end{equation*}

Dividing by $\frac{\delta^2}{2}$.

\begin{equation*}
    0 \geq f''(x_{0}) + \mathcal{O}(\delta)
\end{equation*}

\begin{equation*}
    0 \geq \lim_{\delta \to 0} f''(x_{0}) + \mathcal{O}(\delta)
\end{equation*}

\begin{equation*}
    0 \geq f''(x_{0}) \blacksquare
\end{equation*}

\subsection{c)}
Let u be a be a sufficiently smooth function with a maximum in $ \left<x_{0}, y_{0}\right> $.$ u: \mathbb{R}^{2} \to \mathbb{R} $.
Let $f1(x) = u(x,y=y_{0})$ and $f2(y) = u(x=x_{0},y)$. Then
\begin{equation*}
    \frac{d}{dx}f1(x)\bigg|_{x=x_{0}} = 0 = \frac{\partial}{\partial x}f1(x)\bigg|_{x=x_{0},y=y_{0}}
\end{equation*}{}
\begin{equation*}
    \frac{d}{dy}f2(y)\bigg|_{y=y_{0}} = 0 = \frac{\partial}{\partial y}f2(y)\bigg|_{x=x_{0},y=y_{0}}
\end{equation*}{}

Thus $\nabla u = \mathbf{0}$

\subsection{d)}


\newpage
\section{Problem 2}
\subsection{a}
% TODO: Make a picture of the domain

\subsection{c}
Let $\mathcal{L}=-\frac{d^{2}}{d x^{2}}$. Let u be discrite on a grid with a stepsize of $h$.
\begin{align}
    u_i &= u_i\\
    u_{i+1} &= u_i + h * u_i' + h^2/2 * u_i'' + h^3/6 * u_i''' + \mathcal{O}\left(h^4\right)\\
    u_{i-1} &= u_i - h * u_i' + h^2/2 * u_i'' - h^3/6 * u_i''' + \mathcal{O}\left(h^4\right)\\
\end{align}
\begin{align}
    D_x^+ &= \frac{u_{i+1}-u_i}{h} = u_i' + h/2 * u_i'' + h^2/6 * u_i''' + \mathcal{O}\left(h^3\right)\\
    D_x^- &= \frac{u_{i+1}-u_i}{h} = u_i' - h/2 * u_i'' + h^2/6 * u_i''' + \mathcal{O}\left(h^3\right)\\
    D_{xx} &= -1/h*\left(D_x^+ - D_x^-\right) = u_i'' + \mathcal{O}\left(h^2\right)\\
\end{align}
\begin{equation}
    \mathcal{L} \approx \frac{-u_{i-1}+2u_i-u_{i+1}}{h^2}
\end{equation}

\section{4}
\newpage
\subsection{b}

\begin{proof}
    
Let $L$ be a discretized version of the Laplacian acting on a lexicographig vector of a 2 dimensional rectangular regular grid.
\begin{equation}
    L\mathbf{u} \approx \mathcal{L} u
\end{equation}

As $L$ is a linear operator, we can split it up into 2 seperate operators. Where $D_x'$ and $D_y'$ are the unknown versions of the discretized second derivative operators.

\begin{equation}
    L = D_x' + D_y'
\end{equation}

Lets consider a Matrix $M$.
\begin{equation}
    M = \begin{bmatrix}
        u_{1,1} & u_{1,2} & \dots & u_{1,n_y} \\
        u_{2,1} & u_{2,2} & \dots & u_{2,n_y} \\
        \vdots & \vdots & \ddots & \vdots \\
        u_{n_x,1} & u_{n_x,2} & \dots & u_{n_x,n_y}
    \end{bmatrix}
\end{equation}

Lets consider the discrete 1D versions of the Laplacian as $D_x$ and $D_y$, with sizes $n_x \times n_x$ and $n_y \times n_y$ respectively.
Let $M_x$ and $M_y$ be $nx\times ny$ matrices containing approximated secondtderivatives of $\mathbf{u}$ order similarly as in $M$. It than can easily proven that.
\begin{align}
    D_x M = M_x \\
    M D_y = M_y
\end{align}
Lets introduce $\operatorname{vec}(A)=\left[a_{1,1}, \ldots, a_{m, 1}, a_{1,2}, \ldots, a_{m, 2}, \ldots, a_{1, n}, \ldots, a_{m, n}\right]^{\mathrm{T}}$ as the vectorization of a Matrix. With this Operators, the following properties emerge.
\begin{align}
    \operatorname{vec}(M) &= \mathbf{u} \\
    \operatorname{vec}(M_x) + \operatorname{vec}(M_y) &= L\mathbf{u}
\end{align}
Using $\operatorname{vec}(A B)=\left(I_{m} \otimes A\right) \operatorname{vec}(B)=\left(B^{\mathrm{T}} \otimes I_{k}\right) \operatorname{vec}(A)$ and $D_y^{\mathrm{T}} = D_y$.
\begin{align}
    \operatorname{vec}(D_x M) = (I_{n_y} \otimes D_x)\operatorname{vec}(M) = \operatorname{vec}(M_x)\\
    \operatorname{vec}(M D_y) = (D_y \otimes I_{n_x})\operatorname{vec}(M) = \operatorname{vec}(M_y)
\end{align} 
As $\operatorname{vec}(M) = \mathbf{u}$, $D_x' \mathbf{u} = \operatorname{vec}(M_x)$ and $D_y' \mathbf{u} = \operatorname{vec}(M_y)$.
\begin{align}
    (I_{n_y} \otimes D_x) = D_x' \\
    (D_y \otimes I_{n_x}) = D_y'
\end{align}
Thus $(I_{n_y} \otimes D_x) + (D_y \otimes I_{n_x}) = L$
\end{proof}
%\end{multicols}
\end{document}