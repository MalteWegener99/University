\documentclass{article}
\usepackage{ amssymb }
\usepackage{amsmath}
\usepackage{mathtools}

\begin{document}
\title{Quiz3}
\author{Malte Wegener (4672194)}

\maketitle

\section{Question 1}
MPI(Message Passage Interface) is parallel Programming Model in which there is no global shared memory. Each Processor has its own private memory. If one Procesor needs data from a different Processor, a message has to be send containing the data. There are 4 modes of messaging in MPI.
\begin{itemize}
    \item Standard Send (MPI\_SEND). In this mode, the Implementation decides whether to use synchronous or buffered mode on the fly
    \item Synchronous Send (MPI\_SSEND). In this mode, the send halts the process until a mathcing recieve is available
    \item Buffered Send (MPI\_BSEND). In this mode, the message is passed into a buffer and then send if a matching recieve is available.
    \item Ready Send (MPI\_RSEND). In this mode the message is send regardless if there is a recieve available, but only suceeds if there is a recieve available. There is no reason to ever use this.
\end{itemize}

\section{Question 2}
As the serial execution time is $T^{\star}=n^3$ and $T_p(p) = \frac{n^3}{p}+p^{1.5}n$. The efficiency is given by $E_p = \frac{T^{\star}}{p*T_p(p)}$. In order to keep the efficiency equal when doubling the number of processors, n has to be increased.

\begin{align}
    \frac{1+p^{2.5}}{n_1^2} = \frac{1+(2p)^{2.5}}{n_2^2}\\
    \frac{n_2}{n_1} = \sqrt{\frac{1+(2p)^{2.5}}{1+p^{2.5}}}
\end{align}
Thus the increase in n has to follw the the previous relation when efficiency should stay the same. In the limit as $n \to \infty$ this approaches 2.387.

When keeping execution time constant when increasing n by increasing p the following relation emerges
\begin{align}
\frac{n_1^3}{p_1}+p_1^{1.5}n_1 = \frac{n_2^3}{p_2}+p_2^{1.5}n_2
\end{align}
However when p is increased this will decrease efficiency of the program.

\section{Question 3}
The data locality ratio is the ratio between the required time for computations between 2 communications and the time required for communication. A higher ratio is better for program performance as it indicates that more time is actually spent computing rather than communicating what to compute.


\end{document}