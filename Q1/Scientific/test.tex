\documentclass{article}
\usepackage{listings}

\begin{document}
\title{Quiz1}
\author{Malte Wegener (4672194)}

\maketitle

\section{Question 1}
A Process is a completely isolated program with its own memory, which is not shared with any other Process. A thread on the other hand exist in a Process and shares memory with other threads of the same Process. The creation of a thread is faster than the creation of a Process. The communication between Processes is also more complicated in comparison to the communication between threads.

\section{Question 2}
\begin{lstlisting}
#pragma omp parallel for
for (i=0; i<n/2; i++)
    a[i]=a[i+1]+a[2*i];
\end{lstlisting}
This code is not valid OMP code, as it is anti dependent. Changing the code to output into a different array, will make the code work in parallel. And adding a shared directive to make sure no unnecessary copies of the arrays are made.
\begin{lstlisting}
    int b[n/2];
    #pragma omp parallel for shared(b, a)
    for (i=0; i<n/2; i++)
        b[i]=a[i+1]+a[2*i];
    \end{lstlisting}

\section{Question 3}
A critical section in OMP denotes a region, which can only be executed by one thread at a time. While a barrier is used to wait for all threads to reach the barrier before continuing execution. Therefor a barrier is only passed as soon as all threads arrived at the barrier.


\end{document}
